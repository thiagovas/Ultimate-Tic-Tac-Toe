\documentclass[10pt]{article}


\usepackage{amsmath}
\usepackage{amssymb}
\usepackage{array}
\usepackage[english]{babel}
\usepackage{blindtext}
\usepackage{booktabs}
\usepackage{ctable}
\usepackage{enumitem}
\usepackage{fancyhdr}
\usepackage{float}
\usepackage[a4paper, margin=0.5in]{geometry}
\usepackage[utf8x]{inputenc}
\usepackage{graphicx}
\usepackage{mathrsfs}
\usepackage{placeins}
\usepackage{rotating}
\usepackage{ragged2e}
\usepackage{relsize}
\usepackage{tabularx}
\usepackage{url}
\usepackage{xcolor,colortbl}



% Quotes
\newcommand{\qn}[1]{``#1''}



\title{\textbf{TODO: WRITE A NICE TITLE!}}
\author{Thiago V. de A. Silva\\2017719891}
\date{\today}
\begin{document}

\maketitle

\section{Introduction}

TODO: Talk a little bit about the classic tic-tac-toe, who invented, a little bit of history, that kind of stuff.\\

Then, talk a little bit about the ultimate version of the game, why it's interesting to study it, and stuff like that.
Remember: $3^81$.

\section{Game Rules}
TODO: Talk about the rules of the game



\section{Related Work}
Talk a little bit about the related works.\\
That paper that the authors showed the equivalency classes of the board.\\
There is also the AI's available online.\\
And maybe more...



\section{Payoff Table}
Probably I'm going to define more than one payoff table.\\
Talk about the tables, why I chose the payoffs described in which one of them.



\section{AI}
\subsection{Alpha-Beta Prunning v1}
With the first function for the $A^*$.

\subsection{Alpha-Beta Prunning v2}
With the second function I defined for the $A^*$.




\section{Experiments}
Show the experiments comparing the results 



\section{Future Work}
For now, there is just the monte carlo tree search.\\
Maybe create more payoff tables.



\section{Conclusion}



\end{document}